\documentclass{letter}
\usepackage{hyperref}
\usepackage[utf8]{inputenc}
\usepackage[textwidth=6in,textheight=14in]{geometry}
\signature{Jonathan Ström}
\address{School of Engineering \\ Brown University \\ Providence 02912}
\begin{document}

\begin{letter}{Editorial Department of Building and Environment}
\opening{Dear Sir or Madam:}

I am pleased to submit this manuscript entitled "Sorption Phenomena In Transient Vapor Intrusion Scenarios" for review and possible publication in Building and Environment.
This is original work and has not been previously published by any other journal or conference proceedings; the submission declaration has been complied with.
We have no interests to declare.

% What we do
In this work, we measured the sorption of trichloroethylene (TCE), a contaminant of concern in vapor intrusion, on a variety of materials commonly found in a house, as well as on soil.
We used these data in a three-dimensional finite-element model to explore how contaminant sorption affects contaminant transport in soils and the indoor environment, and how the efficacy of mitigation systems may be impacted by desorption from indoor materials.

% What are the results and why should you care
We believe that the following conclusions contribute to the state-of-the-art:
\begin{itemize}
  \item Different materials can have dramatically different sorptive capacities, with some, like cinderblock, having the capacity to hold 41,000 times more TCE than a comparable volume of contaminated air.
  \item Contaminant transport from the sorbed state can be slow, which can influence timescales in indoor contaminant concentration response.
  \item After a mitigation system has been installed, desorption of TCE from indoor materials can increase the time for a certain reduction in indoor contaminant concentration to occur from a matter of hours to weeks.
\end{itemize}
We expect the topics explored and conclusions reached in our work will be of great interest to your readership.

Thank you for considering this manuscript for publication.
\closing{Sincerely,}

\end{letter}
\end{document}
