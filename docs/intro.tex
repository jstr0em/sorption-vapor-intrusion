\begin{comments}

What is the message of the paper?

Sorbed contaminants can significantly delay changes in concentration in the indoor air and the soil-gas depending on the particular soil and/or indoor materials found.
This has consequences if one is for instance interested in mitigating or remediating a VI site as sorption may significantly impede this effort.
It also has consequences for rooting out indoor contaminant sources as:
1. Even after ventilation and/or removing potential indoor sources, there may still be contaminant vapors being released from various materials.
2. It may decrease the effectiveness of applying the CPM.

What is the new result/contribution that you want to describe?

This study presents some new sorption information for TCE and runs never-done-before simulations that investigate the potential role of sorption in VI and VI investigations.

What do you want to convince people of?

1. Take indoor materials into account and perhaps removing or covering up exposed materials that have a high sorption capacity.
2. Take it into consideration that contaminants vapors may emanate from soils for a long time, since they potentially have such a large sorption capacity - almost acting a source in of themselves. E.g. that remediation or mitigation effort may be impeded by this.
3. Perhaps desorbing soil/indoor material samples to determine how significant sorption might be warranted.

\end{comments}

\section{Introduction}\label{sec:intro}

% Attention grabbing intro
Many vapor intrusion (VI) contaminants has the capacity to sorb onto soil and various common indoor materials, but the role and more importantly - the consequences of these sorption processes in VI are poorly understood.
The migration of contaminant vapors from its source into the affected building and potential indoor sources are usually the prime concern in VI investigations.
Rarely is the sorbed contaminant vapors in the soil or indoor considered in an investigation, but these may potentially act as a capacitor, storing and releasing contaminant vapors in response to a change in contaminant concentration.
Consequently, contaminant vapors may be much more persistent at a site that has undergone remediation, reduce the effectiveness of mitigation systems, or impede site investigations.\par

% Some elaboration on these consequences? E.g. cover:
% - EPA site investigation recommendations
% - CPM and potential issues
% - Remediation (are there any examples of vapor being more persistent?)
% Separate paragraph for each?

% Background on TCE and sorption capacity on materials (Shuai's part). Cover:
% - Which contaminants/material pairs have been investigated? What has been found?
% - Why do we pick TCE and our materials?

% Why use modeling for this? Cover:
% - Briefly why we wish to model
% - Sorption in other models and their applications (or lack thereof I should say)

% Final paragraph - outlining what we will do:
